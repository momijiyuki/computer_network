\begin{table}[htbp]
  \centering
  \begin{tikzpicture}
      % 縦線
      \draw (0,2.5) -- (0,3.2);
      \draw (6,2.5) -- (6,3.2);

      % 32ビット矢印と表示
      \draw[<-] (0,2.85) -- (2.2,2.85);
      \node at (3,2.85) {32ビット};
      \draw[->] (3.8,2.85) -- (6,2.85);

      % 表の枠線
      \draw (0,2.5) -- (6,2.5);
      \draw (0,1.83) -- (6,1.83);
      \draw (0,1.17) -- (6,1.17);
      \draw (0,0) -- (6,0);
      \draw (0,0) -- (0,2.5);
      \draw (6,0) -- (6,2.5);
      \draw (3,2.5) -- (3,1.17);

      % テキスト
      \node at (1.5,2.17) {始点ポート番号};
      \node at (4.5,2.17) {終点ポート番号};
      \node at (1.5,1.5) {セグメントの長さ};
      \node at (4.5,1.5) {チェックサム};
      \node[above] at (3,0.58) {アプリケーション層メッセージ};
    \end{tikzpicture}
\end{table}
