\section{コンピュータネットワーク概観}

\subsection{コンピュータネットワークの構成}

\subsubsection{インターネットの構成}

\begin{itemize}
  \item エンドシステム (end system)
  \begin{itemize}
    \item ネットワークに接続しているコンピュータ
    \item ホストともいう
    \item クライアントとサーバに分類されることが多い
  \end{itemize}
  \item クライアント\\
    サービス要求を出す
  \item サーバ\\
    サービス要求を待ち受ける
  \item 通信リンク\\
    光ファイバ、電波、同軸ケーブル、etc...
  \item ルータ\\
    パケットの中継装置
  \item \textcolor{cyan}{ここまでがモノ}
  \item プロトコル\\
    二つ以上の通信エンティティ間でやりとりされるメッセージの形式と順序などを取り決める規約\\
    \textcolor{cyan}{ネットワークの機器間でのやり取りにおけるルール}
\end{itemize}

\newpage
\subsubsection{通信サービス}

エンドシステム間で情報をやり取りするための仕掛け
\begin{itemize}
  \item \textcolor{orange}{コネクション指向型サービス}\\
    クライアントとサーバは、通信を始める前に相互に制御パケットを送信\\
    \textcolor{cyan}{通信前にハンドシェイク}
    \begin{itemize}
      \item \textcolor{orange}{高信頼データ転送}\\
        順序通り、誤りなく伝送
      \item \textcolor{orange}{フロー制御}\\
        受信側のバッファを溢れさせない\\
        \textcolor{cyan}{空きバイト数をフィードバック}
      \item \textcolor{orange}{輻輳制御}\\
        ネットワークの混雑を防ぐ
    \end{itemize}
    代表的なプロトコルは、TCP (Transmission Control Protocol)
  \item \textcolor{orange}{コネクションレス型サービス}\\
    事前の制御パケットのやり取りなしにいきなりデータを送る\\
    代表的なプロトコルは
    \begin{itemize}
      \item UDP (\textcolor{orange}{User Datagram Protocol})
      \item リアルタイムアプリケーション向き
      \item 高い自由度をもつ
    \end{itemize}
    \textcolor{cyan}{高信頼性が無駄なときや、トランスポート層をアプリケーションからいじることができる(TCPをいじりたければOSアップデートが必要)}
\end{itemize}

\newpage
\subsection{ネットワークコア}
エンドシステム間の相互接続を担うルータ群


\subsubsection{回線交換}
エンドシステム間の通信のために経路に沿った通信資源(バッファや帯域の一部)をセッション中常時常時占有

\textcolor{cyan}{予め通信経路を予約して占有}


\subsubsection{パケット交換}
パケット(oacket)
\begin{itemize}
  \item アプリケーションレベルのメッセージを分割したもの
  \item ネットワークコアでの伝送単位
\end{itemize}
\underline{蓄積交換伝送}
\begin{itemize}
  \item 各ルータで到着したパケットを一旦、バッファへ格納
  \item 予め定められた順序(先着順など)に従い、順次パケットを伝送
\end{itemize}

\subsubsection{回線交換 vs パケット交換}

パケット交換の長所
\begin{itemize}
  \item 伝送容量を効率的に利用可能
  \item 実装が容易 \textcolor{cyan}{ルータに送るだけ}
\end{itemize}

回線交換の長所
\begin{itemize}
  \item 通信品質が安定、リアルタイムアプリケーション向き\\
  \textcolor{cyan}{通信を占有するため、安定性が求められるとき}
\end{itemize}

\begin{table}[h]
  \centering
  \caption{回線交換とパケット交換の長所}
  \label{tab:packet}
  \begin{tabular}{c|l|l}\hline
    & 回線交換 & パケット交換\\\hline
    長所
    & \begin{tabular}{c}
        通信品質が安定\\
        リアルタイムアプリケーション向き
      \end{tabular}
    & \begin{tabular}{c}
      伝送容量を効率的に利用可能\\
      実装が容易
    \end{tabular}\\\hline
  \end{tabular}
\end{table}
