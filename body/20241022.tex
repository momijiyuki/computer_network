\section{アプリケーション層プロトコル}

\subsection{基礎}
\subsubsection{アプリケーション層プロトコル}

\begin{itemize}
  \item ネットワークアプリケーション: ネットワークの使用を前提とするソフトウェア
  \begin{itemize}
    \item[例)] Webアプリケーションは、ブラウザ(Chrome, Safariなど) とWebサーバ (Apacheなど) の複数のソフトウェアから構成
    \item[] \textcolor{cyan}{Webサーバは他にも nginxとか}
  \end{itemize}
\end{itemize}

\begin{itemize}
  \item アプリケーション層プロトコルは。メッセージ交換方法やメッセージのタイプ・フォーマットを規定
  \begin{itemize}
    \item[例)] webでは HTTP を使用
  \end{itemize}
  \item ネットワークアプリケーションは一般にクライアントとサーバの両方の側面をもつ (WebブラウザとWebサーバの関係)
  \item ネットワークを介した \textcolor{orange}{プロセス間通信}
  \item[] \textcolor{cyan}{プロセス間通信は実際には1つ下の層であるトランスポート層が提供 (自由度が高いらしい)}
  \begin{itemize}
    \item 二つのエンドシステムにあるプロセス間でネットワークを介してやり取り
    \item ソケット (Socket) と呼ばれる仮想的なインターフェースを用いる
    \begin{itemize}
      \item ソケットはAPI (Application Programming Interface) の一種
      \item トランスポート層とのインターフェースの役割
    \end{itemize}
  \end{itemize}
  % 黒板の改ページ
  \item コネクション (フロー) の単位
  \begin{itemize}
    \item[※] アプリケーション層が割り当てる情報
    \begin{itemize}
      \item 送受信ホストの \undercolor{IPアドレス}  % 1, 2
      \item プロセスの送受信 \undercolor{ポート番号} % 3, 4
      \begin{itemize}
        \item[$\rightarrow$] ホストのどのプロセスかを特定する番号
      \end{itemize}
      \item サーバ側プロセスは多くの場合ポート番号が固定
      \begin{itemize}
        \item 1023番まではwell-knownポートとして予約 (HTTP:80. SMTP:25, DNS:53 など)
      \end{itemize}
    \end{itemize}
  \end{itemize}
  \item トランスポート層プロトコル \undercolor{(TCP, UDP)}  % 5
  \item[] \textcolor{cyan}{同じポート番号でもプロトコルが違えば使用可能}
\end{itemize}
% 1,2,3,4,5 をまとめてなんか言ってた
% 5個情報送るよって話

\newpage
\subsubsection{トランスポート層プロトコルから提供されるサービス}

アプリケーション製作者は、「データ損失」「帯域幅」「遅延」の観点から選択

\begin{itemize}
  \item TCPサービス
  \begin{itemize}
    \item コネクション指向型: 全2重コネクション。ハンドシェイクを待ってから通信
    \item 高信頼データ伝送: 誤りのない、正しい順序のデータ送信を保証
    \item 輻輳制御: ネットワークの混雑に合わせて伝送レートを変更
  \end{itemize}
  \item UDPサービス
  \begin{itemize}
    \item 最小限のデータ伝送: メッセージの到達性や正確性は保証しない
    \item コネクションレス型: ハンドシェイクは行わない
    \item 輻輳制御なし: 伝送レートは始点プロセスで指定
  \end{itemize}
\end{itemize}

\subsubsection{アプリケーション層プロトコルの分類}

プル型、プッシュ型
\begin{itemize}
  \item プル型プロトコル
  \begin{itemize}
    \item HTTP, FTPなど
    \item ユーザが必要に応じて情報を引き出す (サーバが情報通信)
    \item 情報取得を希望するクライアントがコネクションを開始
    \end{itemize}
  \item プッシュ型プロトコル
  \begin{itemize}
    \item SMTPなど: (メールを送る側が相手へ情報送信)
    \item 情報発信する側がコネクションを開始
  \end{itemize}
\end{itemize}

個別帯域、共通帯域
\begin{itemize}
  \item 個別帯域 (out-of-band) 方式
  \begin{itemize}
    \item 制御用とデータ用で別のコネクションを使用\\
    \textcolor{cyan}{大きなファイルのアップロードの際に制御情報を別ルートで送ることができる}
    \item 代表例はFTP \textcolor{cyan}{Webページの更新など}
  \end{itemize}
  \item 共通帯域 (in-band 方式)
  \begin{itemize}
    \item 制御用とデータ用で同じコネクションを使用
      \textcolor{cyan}{(TCPを想定)}
    \item 代表例はHTTP
  \end{itemize}
\end{itemize}

個別帯域の例
\begin{itemize}
  \item 個別帯域の考え方は色々な場合に存在
  \item OpenFlow
  \item 5GのC/U 分離 (フェムトセル) \textcolor{cyan}{control, user \url{https://jirei.bzlog.jp/5g/information_14/}}
  \begin{itemize}
    \item マクロセルで制御信号をやり取り
    \item フェムトセルで高速データ転送
  \end{itemize}
\end{itemize}
