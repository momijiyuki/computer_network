\subsection{DNS (Domain Name System)}

\begin{itemize}
  \item ネームサーバによる階層的データベース
  \item ホストとネームサーバ間の通信を担うアプリケーション層プロトコル
\end{itemize}

\subsubsection{提供されるサービス}

\begin{itemize}
  \item ホスト名をIPアドレスに変換 (\textbf{名前決定})
  \item ホスト名の別名 (alias: エイリアス) の登録
  \item メールサーバ名の別名登録
\end{itemize}

\begin{itemize}
  \item DNSラウンドロビン: 負荷分散
  \begin{itemize}
    \item 公式ホスト名と複数のミラーサーバのIPアドレスの集合を対応づけ問い合わせ毎にIPアドレスをシフトして通知\\
      (クライアントはリストの一番上を使う)
  \end{itemize}
\end{itemize}

\subsubsection{DNSの仕組み}

\begin{itemize}
  \item \textbf{なぜ集中型でないか}
  \item 分散型モデル(ネームサーバの分類)
  \begin{itemize}
    \item 障害発生時の信頼性欠如(single point of failure)
    \item サーバへのトラヒック集中
    \item 遠隔地との通信遅延が大きい
    \item 保守更新の煩雑さ
  \end{itemize}
  \item 分散型モデル(ネームサーバの分類)
  \begin{itemize}
    \item ルートDNSサーバ(世界で13系統)
    \begin{itemize}
      \item トップレベルDNSサーバのIPアドレスを把握
    \end{itemize}
    \item トップレベルドメイン(top-level domain)DNSサーバ
    \begin{itemize}
      \item jp, com, eduなどのトップレベルドメインを管理
    \end{itemize}
    \item 権威(Authoritative)DNSサーバ
    \begin{itemize}
      \item ホスト名(ドメイン名)とIPアドレスの変換機能を持つ
      \item 各ホストは必ず一つ以上の権威DNSサーバに登録
    \end{itemize}
    \item[※] ローカルDNSサーバ:DNSキャッシュサーバ、DNSリゾルバ
    \begin{itemize}
      \item ISPや各企業が個別に持つ
      \item ユーザは通常、ローカルDNSサーバに名前解決を依頼
    \end{itemize}
  \end{itemize}
  \item DNS反復クエリとDNS再帰クエリ
  \begin{itemize}
    \item DNS反復クエリ
    \begin{itemize}
      \item 問い合わせたネームサーバへの名前の解決を依頼
    \end{itemize}
    \item DNS再帰クエリ
    \begin{itemize}
      \item 問い合わせるべきネームサーバの情報を要求
    \end{itemize}
  \end{itemize}
\end{itemize}

\begin{itemize}
  \item ローカルDNSサーバはホストへ転送した情報をキャッシュする:DNSキャッシュ
  \item DNSレコード
  \begin{itemize}
    \item リソースレコードを格納する分散データベース
    \item (name, value, type, ttl)の4フィールドで構成
    \item Type A レコード
    \begin{itemize}
      \item nameはホスト名, valueはIPadoresu (IPv6の場合、AAAAレコード)
    \end{itemize}
    \item Type NS レコード
    \begin{itemize}
      \item nameはドメイン名, valueはその権威DNSサーバホスト名
    \end{itemize}
    \item Type CNAMEレ コード
    \begin{itemize}
      \item ホスト名の別名
    \end{itemize}
    \item Type MX レコード
    \begin{itemize}
      \item nameに関するメルサーバ情報を格納
    \end{itemize}
  \end{itemize}
  \item TTLは秒単位で記載
  \begin{itemize}
    \item この期間を過ぎるとキャッシュから削除
  \end{itemize}
\end{itemize}

\subsubsection{DNSサーバへの攻撃}
\begin{itemize}
  \item DNSキャッシュポイズニング
  \begin{enumerate}
    \item 攻撃者がDNSキャッシュサーバへ名前解決を依頼
    \item DNSキャッシュサーバは再帰的に問合せ
    \item 攻撃者は即座に、偽装したUDPパケットで返答
    \item キャッシュサーバは、誤ったドメイン情報を記憶
  \end{enumerate}
  \item[※] 攻撃要件
  \begin{enumerate}
    \item 発信元DNSサーバのIPアドレスに偽装
    \item 問い合わせと同じUDPポート番号を使用
    \item 問い合わせと同じID(6/bit)を使用
    \item 本来のDNSサーバより先に応答
    \begin{itemize}
      \item さらに、攻撃のチャンスはTTL毎に一度だけ
    \end{itemize}
  \end{enumerate}
\end{itemize}

\newpage
\subsection{コンテンツ分散}

\begin{itemize}
  \item コンテンツが一つのサーバ上のみにあると
  \begin{itemize}
    \item 伝送遅延が増大
    \item サーバへの負担が大きい
  \end{itemize}
  \item コンテンツ分散
  \begin{itemize}
    \item コンテンツをインターネット上の複数サーバにコピー
    \item ユーザに最も迅速に配送できるサーバを見つけ、コンテンツを配信
  \end{itemize}
\end{itemize}

\subsubsection{CDN (Content Distribution Network)}
\begin{itemize}
  \item 90年代後半から登場したビジネスモデル
  \item CDNサービス企業は、インターネット上に多数のサーバ(CDNサーバ)を配置
  \begin{itemize}
    \item (NetfrixやYoutubeは自社でCDNサーバを設置)
    \item 顧客のコンテンツをCDNサーバにコピー(各更新毎に)
    \item ユーザのコンテンツを要求に対して、適切なCDNサーバを選択
  \end{itemize}
  \item[$\rightarrow$] \textbf{ユーザごとにDNSの返答を変える}
\end{itemize}
